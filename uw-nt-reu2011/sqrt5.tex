\documentclass{article} 

\bibliographystyle{amsalpha} 

\voffset=-0.2\textheight  \textheight=1.4\textheight
%\hoffset=-0.2\textwidth \textwidth=1.4\textwidth
%\voffset=-0.05\textheight \textheight=1.1\textheight
\hoffset=-0.05\textwidth \textwidth=1.1\textwidth



% macros.tex
\usepackage{amsmath}
\usepackage{amsfonts}
\usepackage{amssymb}
\usepackage{amsthm}

\usepackage{url}


% You change everything, by adding \usepackage{times} to the document
% Preamble. Now all the roman letters will be set in times and all the
% sans serif stuff will be set in Helvetica. If you don't like times,
% you can try the packages: palatcm, charter, helvet, palatino, avant,
% newcent and bookman
% If you want to change explicitly to a certain font, use the command
% \fontfamily{XYZ}\selectfont whereby XYZ can be set to: pag for Adobe
% AvantGarde, pbk for Adobe Bookman, pcr for Adobe Courier, phv for
% Adobe Helvetica, pnc for Adobe NewCenturySchoolbook, ppl for Adobe
% Palatino, ptm for Adobe Times Roman, pzc for Adobe ZapfChancery
\newcommand{\courier}{\fontfamily{pcr}\selectfont}



\newcommand{\edit}[1]{\footnote{[[#1]]}\marginpar{\hfill {\sf[[\thefootnote]]}}}
%\newcommand{\edit}[1]{{\sl\small [[Todo: #1]]}}


%\author{William~A. Stein}

\newcommand{\Hbar}{\overline{H}}

\newcommand{\myhead}[3]{
\par\noindent
{Version #2}
\vspace{10ex}
\par\noindent
{\bf \LARGE #1}\\
\vspace{3ex}
\par\noindent
{\large W.\thinspace{}A. Stein}\\
{\small Department of Mathematics, Harvard University}\vspace{1ex}\\
#3     
\vspace{2ex}\par
}

\newcommand{\myheadauth}[3]{
\par\noindent
{Version #2}
\vspace{10ex}
\par\noindent
{\bf \LARGE #1}\\
\vspace{3ex}
\par\noindent
#3     
\vspace{5ex}\par
}

\usepackage{xspace}  % to allow for text macros that don't eat space 
\newcommand{\SAGE}{{\sf Sage}\xspace}
\newcommand{\sage}{\SAGE}
\newcommand{\gzero}{\Gamma_0(N)}
\newcommand{\esM}{\overline{\sM}}
\newcommand{\sM}{\boldsymbol{\mathcal{M}}}
\newcommand{\sS}{\boldsymbol{\mathcal{S}}}
\newcommand{\sB}{\boldsymbol{\mathcal{B}}}       
\newcommand{\bA}{\mathbb{A}}
\newcommand{\cK}{\mathcal{K}}
\newcommand{\Adual}{A^{\vee}}
\newcommand{\Bdual}{B^{\vee}}
\newcommand{\kr}[2]{\left(\frac{#1}{#2}\right)}

\newcommand{\defn}[1]{{\em #1}}
\newcommand{\solution}[1]{\vspace{1em}%
  \par\noindent{\bf Solution #1.} }
\newcommand{\todo}[1]{\noindent$\bullet$ {\small \textsf{#1}} $\bullet$\\}
\newcommand{\done}[1]{\noindent {\small \textsf{Done: #1}}\\}
\newcommand{\danger}[1]{\marginpar{\small \textsl{#1}}}
\renewcommand{\div}{\mbox{\rm div}}
\DeclareMathOperator{\GCD}{GCD}
\DeclareMathOperator{\CH}{CH}
\DeclareMathOperator{\sss}{ss}
\renewcommand{\ss}{\sss}
\DeclareMathOperator{\red}{red}
\DeclareMathOperator{\sat}{sat}
\DeclareMathOperator{\xgcd}{xgcd}
\DeclareMathOperator{\Kol}{Kol}
\DeclareMathOperator{\can}{can}
\DeclareMathOperator{\Cl}{Cl}
\DeclareMathOperator{\Mod}{Mod}
\DeclareMathOperator{\mods}{\textbf{mods}}
\DeclareMathOperator{\chr}{char}
\DeclareMathOperator{\charpoly}{charpoly}
\DeclareMathOperator{\cris}{cris}
\DeclareMathOperator{\dR}{dR}
\DeclareMathOperator{\Fil}{Fil}
\DeclareMathOperator{\ind}{ind}
\DeclareMathOperator{\im}{im}
\DeclareMathOperator{\oo}{\infty}
\DeclareMathOperator{\abs}{abs}
\DeclareMathOperator{\lcm}{lcm}
\DeclareMathOperator{\cores}{cores}
\DeclareMathOperator{\coker}{coker}
\DeclareMathOperator{\image}{image}
\DeclareMathOperator{\prt}{part}
\DeclareMathOperator{\proj}{proj}
\DeclareMathOperator{\Br}{Br}
\DeclareMathOperator{\Ann}{Ann}
\DeclareMathOperator{\End}{End}
\DeclareMathOperator{\Tan}{Tan}
\DeclareMathOperator{\Eis}{Eis}
\newcommand{\unity}{\mathbb{1}}
\DeclareMathOperator{\Pic}{Pic}
\DeclareMathOperator{\Tate}{Tate}
\DeclareMathOperator{\Vol}{Vol}
\DeclareMathOperator{\Vis}{Vis}
\DeclareMathOperator{\Reg}{Reg}
%\DeclareMathOperator{\myRes}{Res}
%\newcommand{\Res}{\myRes}
\DeclareMathOperator{\Res}{Res}
\newcommand{\an}{{\rm an}}
\DeclareMathOperator{\rank}{rank}
\DeclareMathOperator{\Sel}{Sel}
\DeclareMathOperator{\Mat}{Mat}
\DeclareMathOperator{\BSD}{BSD}
\DeclareMathOperator{\id}{id}
\DeclareMathOperator{\dz}{dz}
%\DeclareMathOperator{\Re}{Re}
\renewcommand{\Re}{\mbox{\rm Re}}
\DeclareMathOperator{\Imm}{Im}
\renewcommand{\Im}{\Imm}
\DeclareMathOperator{\Selmer}{Selmer}
\newcommand{\pfSel}{\widehat{\Sel}}
\newcommand{\qe}{\stackrel{\mbox{\tiny ?}}{=}}
\newcommand{\isog}{\simeq}
\newcommand{\e}{\mathbf{e}}
\newcommand{\bN}{\mathbf{N}}

% ---- SHA ----
\DeclareFontEncoding{OT2}{}{} % to enable usage of cyrillic fonts
  \newcommand{\textcyr}[1]{%
    {\fontencoding{OT2}\fontfamily{wncyr}\fontseries{m}\fontshape{n}%
     \selectfont #1}}
\newcommand{\Sha}{{\mbox{\textcyr{Sh}}}}

%\font\cyr=wncyr10 scaled \magstep 1
%\font\cyr=wncyr10

%\newcommand{\Sha}{{\cyr X}}
\newcommand{\Shaan}{\Sha_{\mbox{\tiny \rm an}}}
\newcommand{\TS}{Shafarevich-Tate group}

\newcommand{\Gam}{\Gamma}
\newcommand{\X}{\mathcal{X}}
\newcommand{\cH}{\mathcal{H}}
\newcommand{\cA}{\mathcal{A}}
\newcommand{\cF}{\mathcal{F}}
\newcommand{\cG}{\mathcal{G}}
\newcommand{\cJ}{\mathcal{J}}
\newcommand{\cL}{\mathcal{L}}
\newcommand{\cV}{\mathcal{V}}
\newcommand{\cO}{\mathcal{O}}
\newcommand{\cQ}{\mathcal{Q}}
\newcommand{\cX}{\mathcal{X}}
\newcommand{\ds}{\displaystyle}
\newcommand{\M}{\mathcal{M}}
\newcommand{\E}{\mathcal{E}}
\renewcommand{\L}{\mathcal{L}}
\newcommand{\J}{\mathcal{J}}
\DeclareMathOperator{\new}{new}
\DeclareMathOperator{\Morph}{Morph}
\DeclareMathOperator{\old}{old}
\DeclareMathOperator{\Sym}{Sym}
\DeclareMathOperator{\Symb}{Symb}
%\newcommand{\Sym}{\mathcal{S}{\rm ym}}
\newcommand{\dw}{\delta(w)} 
\newcommand{\dwh}{\widehat{\delta(w)}}      
\newcommand{\dlwh}{\widehat{\delta_\l(w)}} 
\newcommand{\dash}{-\!\!\!\!-\!\!\!\!-\!\!\!\!-} 
\DeclareMathOperator{\tor}{tor}  
\newcommand{\Frobl}{\Frob_{\ell}}
\newcommand{\tE}{\tilde{E}}
\renewcommand{\l}{\ell}
\renewcommand{\t}{\tau}
\DeclareMathOperator{\cond}{cond}
\DeclareMathOperator{\Spec}{Spec}
\DeclareMathOperator{\Div}{Div}
\DeclareMathOperator{\Jac}{Jac}
\DeclareMathOperator{\res}{res}
\DeclareMathOperator{\Ker}{Ker}
\DeclareMathOperator{\Coker}{Coker}
\DeclareMathOperator{\sep}{sep}
\DeclareMathOperator{\sign}{sign}
\DeclareMathOperator{\unr}{unr}
\newcommand{\N}{\mathcal{N}}
\newcommand{\U}{\mathcal{U}}
\newcommand{\Kbar}{\overline{K}}
\newcommand{\Lbar}{\overline{L}}
\newcommand{\gammabar}{\overline{\gamma}}
\newcommand{\q}{\mathbf{q}}
%\renewcommand{\star}{\times}
\newcommand{\gM}{\mathfrak{M}}
\newcommand{\gA}{\mathfrak{A}}
\newcommand{\gP}{\mathfrak{P}}
\newcommand{\bmu}{\boldsymbol{\mu}}
\newcommand{\union}{\cup}
\newcommand{\Tl}{T_{\ell}}
\newcommand{\into}{\rightarrow}
\newcommand{\onto}{\twoheadrightarrow}%  Surjection arrow

\newcommand{\meet}{\cap}
\newcommand{\cross}{\times}
\DeclareMathOperator{\md}{mod}
\DeclareMathOperator{\toric}{toric}
\DeclareMathOperator{\tors}{tors}
\DeclareMathOperator{\Frac}{Frac}
\DeclareMathOperator{\corank}{corank}
\newcommand{\rb}{\overline{\rho}}
\newcommand{\ra}{\rightarrow}
\newcommand{\xra}[1]{\xrightarrow{#1}}
\newcommand{\hra}{\hookrightarrow}
\newcommand{\la}{\leftarrow}
\newcommand{\lra}{\longrightarrow}
\newcommand{\riso}{\xrightarrow{\sim}}
\newcommand{\da}{\downarrow}
\newcommand{\ua}{\uparrow}
\newcommand{\con}{\equiv}
\newcommand{\Gm}{\mathbb{G}_m}
\newcommand{\pni}{\par\noindent}
\newcommand{\set}[1]{\{#1\}}
\newcommand{\iv}{^{-1}}
\newcommand{\alp}{\alpha}
\newcommand{\bq}{\mathbf{q}}
\newcommand{\cpp}{{\tt C++}}
\newcommand{\tensor}{\otimes}
\newcommand{\bg}{{\tt BruceGenus}}
\newcommand{\abcd}[4]{\left(
        \begin{smallmatrix}#1&#2\\#3&#4\end{smallmatrix}\right)}
\newcommand{\mthree}[9]{\left(
        \begin{matrix}#1&#2&#3\\#4&#5&#6\\#7&#8&#9
        \end{matrix}\right)}
\newcommand{\mtwo}[4]{\left(
        \begin{matrix}#1&#2\\#3&#4
        \end{matrix}\right)}
\newcommand{\vtwo}[2]{\left(
        \begin{matrix}#1\\#2
        \end{matrix}\right)}
\newcommand{\smallmtwo}[4]{\left(
        \begin{smallmatrix}#1&#2\\#3&#4
        \end{smallmatrix}\right)}
\newcommand{\twopii}{2\pi{}i{}}  
\newcommand{\eps}{\varepsilon}
\newcommand{\vphi}{\varphi}
\newcommand{\gp}{\mathfrak{p}}
\newcommand{\W}{\mathcal{W}}
\newcommand{\oz}{\overline{z}}
\newcommand{\Zpstar}{\Zp^{\star}}
\newcommand{\Zhat}{\widehat{\Z}}
\newcommand{\Zbar}{\overline{\Z}}
\newcommand{\Zl}{\Z_{\ell}}
\newcommand{\comment}[1]{}
\newcommand{\Q}{\mathbb{Q}}
\newcommand{\QQ}{\mathbb{Q}}
\newcommand{\GQ}{G_{\Q}}
\newcommand{\R}{\mathbb{R}}
\newcommand{\RR}{\mathbb{R}}
\newcommand{\PP}{\mathbb{P}}
\newcommand{\D}{{\mathbf D}}
\newcommand{\cC}{\mathcal{C}}
\newcommand{\cD}{\mathcal{D}}
\newcommand{\cP}{\mathcal{P}}
\newcommand{\cS}{\mathcal{S}}
\newcommand{\Sbar}{\overline{S}}
\newcommand{\K}{{\mathbb K}}
\newcommand{\C}{\mathbb{C}}
\newcommand{\CC}{\mathbb{C}}
\newcommand{\Cp}{{\mathbb C}_p}
\newcommand{\Sets}{\mbox{\rm\bf Sets}}
\newcommand{\bcC}{\boldsymbol{\mathcal{C}}}
\renewcommand{\P}{\mathbb{P}}
\newcommand{\Qbar}{\overline{\Q}}
\newcommand{\QQbar}{\overline{\Q}}
\newcommand{\kbar}{\overline{k}}
\newcommand{\dual}{\bot}
\newcommand{\T}{\mathbb{T}}
\newcommand{\TT}{\mathbb{T}}
\newcommand{\calT}{\mathcal{T}}
\newcommand{\cT}{\mathcal{T}}
\newcommand{\cbT}{\mathbb{\mathcal{T}}}
\newcommand{\cU}{\mathcal{U}}
\newcommand{\Z}{\mathbb{Z}}
\newcommand{\ZZ}{\mathbb{Z}}
\newcommand{\F}{\mathbb{F}}
\newcommand{\FF}{\mathbb{F}}
\newcommand{\Fl}{\F_{\ell}}
\newcommand{\Fell}{\Fl}
\newcommand{\Flbar}{\overline{\F}_{\ell}}
\newcommand{\Flnu}{\F_{\ell^{\nu}}}
\newcommand{\Fbar}{\overline{\F}}
\newcommand{\Fpbar}{\overline{\F}_p}
\newcommand{\fbar}{\overline{f}}
\newcommand{\Qp}{\Q_p}
\newcommand{\Ql}{\Q_{\ell}}
\newcommand{\Qell}{\Q_{\ell}}
\newcommand{\Qlbar}{\overline{\Q}_{\ell}}
\newcommand{\Qlnr}{\Q_{\ell}^{\text{nr}}}
\newcommand{\Qlur}{\Q_{\ell}^{\text{ur}}}
\newcommand{\Qltm}{\Q_{\ell}^{\text{tame}}}
\newcommand{\Qv}{\Q_v}
\newcommand{\Qpbar}{\Qbar_p}
\newcommand{\Zp}{\Z_p}
\newcommand{\Fp}{\F_p}
\newcommand{\Fq}{\F_q}
\newcommand{\Fqbar}{\overline{\F}_q}
\newcommand{\Ad}{Ad}
\newcommand{\adz}{\Ad^0}
\renewcommand{\O}{\mathcal{O}}
\newcommand{\A}{\mathcal{A}}
\newcommand{\Og}{O_{\gamma}}
\newcommand{\isom}{\cong}
\newcommand{\ncisom}{\approx}   % noncanonical isomorphism
\DeclareMathOperator{\ab}{ab}
\DeclareMathOperator{\alg}{alg}
\DeclareMathOperator{\Aut}{Aut}
\DeclareMathOperator{\Frob}{Frob}
\DeclareMathOperator{\Fr}{Fr}
\DeclareMathOperator{\Ver}{Ver}
\DeclareMathOperator{\Norm}{Norm}
\DeclareMathOperator{\Ind}{Ind}
\DeclareMathOperator{\norm}{norm}
\DeclareMathOperator{\disc}{disc}
\DeclareMathOperator{\ord}{ord}
\DeclareMathOperator{\GL}{GL}
\DeclareMathOperator{\PSL}{PSL}
\DeclareMathOperator{\PGL}{PGL}
\DeclareMathOperator{\Gal}{Gal}
\DeclareMathOperator{\SL}{SL}
\DeclareMathOperator{\SO}{SO}
\DeclareMathOperator{\WC}{WC}
\newcommand{\galq}{\Gal(\Qbar/\Q)}
\newcommand{\rhobar}{\overline{\rho}}
\newcommand{\cM}{\mathcal{M}}
\newcommand{\cB}{\mathcal{B}}
\newcommand{\cE}{\mathcal{E}}
\newcommand{\cR}{\mathcal{R}}
\newcommand{\et}{\text{\rm\'et}}

\newcommand{\sltwoz}{\SL_2(\Z)}
\newcommand{\sltwo}{\SL_2}
\newcommand{\gltwoz}{\GL_2(\Z)}
\newcommand{\mtwoz}{M_2(\Z)}
\newcommand{\gltwoq}{\GL_2(\Q)}
\newcommand{\gltwo}{\GL_2}
\newcommand{\gln}{\GL_n}
\newcommand{\psltwoz}{\PSL_2(\Z)}
\newcommand{\psltwo}{\PSL_2}
\newcommand{\h}{\mathfrak{h}}
\renewcommand{\a}{\mathfrak{a}}
\newcommand{\p}{\mathfrak{p}}
\newcommand{\m}{\mathfrak{m}}
\newcommand{\trho}{\tilde{\rho}}
\newcommand{\rhol}{\rho_{\ell}}
\newcommand{\rhoss}{\rho^{\text{ss}}}
\DeclareMathOperator{\tr}{tr}
\DeclareMathOperator{\order}{order}
\DeclareMathOperator{\ur}{ur}
\DeclareMathOperator{\Tr}{Tr}
\DeclareMathOperator{\Hom}{Hom}
\DeclareMathOperator{\Mor}{Mor}
\DeclareMathOperator{\HH}{H}
\renewcommand{\H}{\HH}
\DeclareMathOperator{\Ext}{Ext}
\DeclareMathOperator{\Tor}{Tor}
\newcommand{\smallzero}{\left(\begin{smallmatrix}0&0\\0&0
                        \end{smallmatrix}\right)}
\newcommand{\smallone}{\left(\begin{smallmatrix}1&0\\0&1
                        \end{smallmatrix}\right)}

\newcommand{\pari}{{\sc Pari}}
\newcommand{\magma}{{\sc Magma}}
\newcommand{\hecke}{{\sc Hecke}}
\newcommand{\lidia}{{\sc LiDIA}}

%%%% Theoremstyles
\theoremstyle{plain}
\newtheorem{theorem}{Theorem}[section]
\newtheorem{proposition}[theorem]{Proposition}
\newtheorem{corollary}[theorem]{Corollary}
\newtheorem{claim}[theorem]{Claim}
\newtheorem{lemma}[theorem]{Lemma}
\newtheorem{hypothesis}[theorem]{Hypothesis}
\newtheorem{conjecture}[theorem]{Conjecture}

\theoremstyle{definition}
\newtheorem{definition}[theorem]{Definition}
\newtheorem{question}[theorem]{Question}
\newtheorem{idea}[theorem]{Idea}
\newtheorem{project}[theorem]{Project}
\newtheorem{problem}[theorem]{Problem}
\newtheorem{openproblem}[theorem]{Open Problem}
\newtheorem{challenge}[theorem]{Challenge}

%\theoremstyle{remark}
\newtheorem{goal}[theorem]{Goal}
\newtheorem{remark}[theorem]{Remark}
\newtheorem{remarks}[theorem]{Remarks}
\newtheorem{example}[theorem]{Example}
\newtheorem{exercise}[theorem]{Exercise}

\numberwithin{equation}{section}
\numberwithin{figure}{section}
\numberwithin{table}{section}


% bulleted list environment
\newenvironment{bulletlist}
   {
      \begin{list}
         {$\bullet$}
         {
            \setlength{\itemsep}{.5ex}
            \setlength{\parsep}{0ex}
            \setlength{\parskip}{0ex}
            \setlength{\topsep}{.5ex}
         }
   }
   {
      \end{list}
   }
%end newenvironment

% bulleted list environment
\newenvironment{dashlist}
   {
      \begin{list}
         {---}
         {
            \setlength{\itemsep}{.5ex}
            \setlength{\parsep}{0ex}
            \setlength{\parskip}{0ex}
            \setlength{\topsep}{.5ex}
         }
   }
   {
      \end{list}
   }
%end newenvironment

% numbered list environment
\newcounter{listnum}
\newenvironment{numlist}
   {
      \begin{list}
            {{\em \thelistnum.}}{
            \usecounter{listnum}
            \setlength{\itemsep}{.5ex}
            \setlength{\parsep}{0ex}
            \setlength{\parskip}{0ex}
            \setlength{\topsep}{.5ex}
         }
   }
   {
      \end{list}
   }
%end newenvironment

\newcommand{\hd}[1]{\vspace{1ex}\noindent{\bf #1} }
\newcommand{\nf}[1]{\underline{#1}} 
\newcommand{\cbar}{\overline{c}}

\DeclareMathOperator{\rad}{rad}

\theoremstyle{definition}
\newtheorem{algor}[theorem]{Algorithm}
\newenvironment{algorithm}[1]{%
\begin{algor}[#1]\index{{\bf Algorithm}!#1}
}%
{\end{algor}}

\newenvironment{steps}%
{\begin{enumerate}\setlength{\itemsep}{0.1ex}}{\end{enumerate}}

\usepackage{color}
\usepackage{listings} 
\lstdefinelanguage{Sage}[]{Python}
{morekeywords={True,False,sage,singular},
sensitive=true}
\lstset{
  showtabs=False,
  showspaces=False,
  showstringspaces=False,
  commentstyle={\ttfamily\color{dredcolor}},
  keywordstyle={\ttfamily\color{dbluecolor}\bfseries},
  stringstyle ={\ttfamily\color{dgraycolor}\bfseries},
  language = Sage,
  basicstyle={\small \ttfamily},
  aboveskip=1em,
  belowskip=1em,
  backgroundcolor=\color{lightyellow},
  frame=single
}
\definecolor{lightyellow}{rgb}{1,1,.86}
\definecolor{dblackcolor}{rgb}{0.0,0.0,0.0}
\definecolor{dbluecolor}{rgb}{.01,.02,0.7}
\definecolor{dredcolor}{rgb}{0.8,0,0}
\definecolor{dgraycolor}{rgb}{0.30,0.3,0.30}
\definecolor{graycolor}{rgb}{0.35,0.35,0.35}
\newcommand{\dblue}{\color{dbluecolor}\bf}
\newcommand{\dred}{\color{dredcolor}\bf}
\newcommand{\dblack}{\color{dblackcolor}\bf}
\newcommand{\gray}{\color{graycolor}}

%%% Local Variables: 
%%% mode: latex
%%% TeX-master: t
%%% End: 







\usepackage{pdfsync}
\usepackage{hyperref}


\title{Elliptic Curves over $\Q(\sqrt{5})$}

\author{Jon Bober, Alyson Deines, Joanna Gaski, Ariah Klages-Mundt,
Ben LeVeque,\\Andrew Ohana, Ashwath Rabindranath, Paul Sharaba, 
William Stein\footnote{The work was supported by NSF 
grants (mainly the FRG)...}\footnote{This is a {\em long} list of authors for a math
paper.  Anybody who for some reason does not feel comfortable
being a co-author, e.g., would rather be moved to acknowledgements, please let William know.}}

\begin{document}
\maketitle

\begin{abstract} 
\end{abstract}

% useful when working on the paper -- remove for publication, probably.
\tableofcontents

\section{Introduction}\label{sec:intro}

Introduction and overview goes here.

\noindent{\bf Acknowledgement: } Richard Taylor, Rado Kirov. 

\section{Sorting and Labeling Curves}\label{sec:sort}
    Our goal in this section is to present a total ordering upon isomorphism
classes of elliptic curves in quadratic fields given a choice of integral
basis. Our method of attack will be to continually refine a partial ordering
until we have a total ordering.

Let $K$ be an abelian field, $G=\Gal(K/\QQ)$, and $\cO_K$ be our ring of
integers. We start by placing a partial ordering upon the elements of $K$.
$\CC$. If $e_k$ is the $k$-th elementary symmetric function, and $|G\alpha|$
is a vector whose components are the absolute value of elements of the orbit
of $\alpha$ under $G$ for some embedding in $\CC$, then we say that $\alpha <
\beta$ if the exists a $k$ such that $e_k(|G\alpha|) < e_k(|G\beta|)$ and
$e_i(|G\alpha|) = e_i(|G\beta|)$ for all $i > k$.\footnote{It should be noted
that the choice of embedding does not affect the ordering, since the values are
permuted by $G$.} 

It is clear that the set of elements that are incomparable with $\alpha$
include $\zeta \sigma\alpha$ for all $\zeta\in\mu_K$ and $\sigma\in G$. In
fact, it can be seen that all incomparable elements are of this form, since if
two elements are incomparable, then $|G\alpha|$ and $|G\beta|$ on all symmetric
functions on $[K:\QQ]$ variables, hence they must be equal. This implies that
$|\beta|=|\sigma\alpha|$ for some $\sigma\in G$, so since $K$ is abelian, we
know there is some $\zeta\in \mu_K$ such that $\beta = \zeta\sigma\alpha$.

This partial ordering naturally gives a partial ordering upon ideals of $O_K$,
by defining $e_k(I) = \min\{e_k(|G\alpha|) : \alpha \in I\}$ and saying that
$I < J$ if there exists a $k$ such that $e_k(I) < e_k(J)$ and $e_i(I) = e_i(J)$
for all $i > k$. In this case we can see that two ideals are incomparable if
and only if one can be obtain by applying a field automorphism to the elements
of the other.

To order between Galois conjugate ideals, we specialize to quadratic fields,
and assume that we have chosen $\{1,\gamma\}$ as our integral basis.
For each ideal $I$ we define $\psi_I:O_K/I \to \ZZ/N(R_I)\ZZ$ to be unique
homomorphism that factors through $O_K/R_I$, where $R_I = I/(I\cap \ZZ)O_K$.
We then refine our previous ordering by saying that $I < J$ if 
$\mods(\psi_I(\gamma)) < \mods(\psi_J(\gamma))$, where $\mods:\ZZ/n\ZZ\to \ZZ$
is the lift that lies in the interval $(-n/2,n/2]$. With this we have a total
ordering upon ideals of $O_K$.




\section{Computing Hilbert Modular Forms}\label{sec:hmf}
    
\subsection{Introduction and Motivation}


\subsection{Quaternion Algebras}

\subsection{Eichler Orders}

\subsection{Class Sets}

\subsection{Hecke Operators}





\section{Finding Curves Associated to Hilbert Modular Forms}\label{sec:finding}
 
\subsection{Naive Search}

\subsection{Naive Search over Curves with Given $a_p$}
    In our attempt to find new elliptic curves over $K=\mathbb{Q}(\sqrt{5})$, we turned to $a_\mathfrak{p}$-values as a major source of information. The $a_\mathfrak{p}$-value of a curve $E$ at a prime $\mathfrak{p} \in K$ can be explicitly given by: $a_\mathfrak{p}=N(\mathfrak{p})+1-\#E(\mathbb{F}_p)$, so it is fairly easy to compute using Sage's built-in point counting methods. It is also referred to as the trace of Frobenius, for in the endomorphism ring $End(E/\mathbb{F}_p)$,
\begin{equation}
Frob_p^2 + a_\mathfrak{p}Frob_p + N(\mathfrak{p}) = 0, \nonumber
\end{equation}
where $Frob_p$ is the Frobenius map sending $(x,y) \mapsto (x^p,y^p) \in E/\mathbb{F}_p$.

Our method for finding an unknown curve $E_{un}$ involves finding all curves in $\mathcal{O}_K/(p)$ which have the correct $a_{\mathfrak{p}_1}$ and $a_{\mathfrak{p}_2}$ values (for $p$ a split prime whose factors $\mathfrak{p}_1$ and $\mathfrak{p}_2$ in $K$ have norm less than 100). In some cases, we also ``combine'' two such rings $\mathcal{O}_K/(p)$ and $\mathcal{O}_K/(q)$ to conduct a more extensive search in $\mathcal{O}_K/(p\cdot q)$. This approach provides congruence conditions on the space of possible candidates for $E_{un}$ and dramatically reduces the number of curves we must search through after lifting each of these candidates to $\mathcal{O}_K$.

\subsubsection{Curves in $\mathcal{O}_K/\mathfrak{p}$}

Let $\mathfrak{p}$ be a prime above the split prime $p$. We know that in characteristic not 2 or 3, we can reduce any elliptic curve to short Weierstrass form (SWF), and since we are only considering primes $\mathfrak{p}$ above split primes (which are always congruent to 1 or 4 modulo 5), we will never run into issues. The first step in our method is to create a dictionary of all nonsingular SWF curves in $\mathcal{O}_K/\mathfrak{p}$, where the keys are $a_\mathfrak{p}$-values and the corresponding entries are the curves which have that $a_\mathfrak{p}$-value. Since $\mathfrak{p}$ is a maximal ideal in $\mathcal{O}_K$, $\mathcal{O}_K/\mathfrak{p}$ is a field of size $p$, so this amounts to finding all SWF curves $E$ with coefficients in $\mathbb{Z}/p\mathbb{Z}$:
\begin{equation}
E: \ \ y^2=x^3+Ax+B, \phantom{asdpoifaofi} A,B \in \mathbb{Z}/p\mathbb{Z} \nonumber
\end{equation}
\noindent with the given $a_\mathfrak{p}$. Using Sage's point-counting functionality, we can easily find the $a_\mathfrak{p}$ value for each of these curves and thus construct our dictionary.

\subsubsection{Curves in $\mathcal{O}_K/(p)$}

Let $p=\mathfrak{p}_1\cdot\mathfrak{p}_2$ in $K$. In order to find curves in $\mathcal{O}_K/(p)$ that have correct $a_\mathfrak{p}$-values at both $\mathfrak{p}_1$ and $\mathfrak{p}_2$, we simply use the Chinese Remainder Theorem on each pair of curves $E_{\mathfrak{p}_1}$ and $E_{\mathfrak{p}_2}$ produced by the method described above (applied to $\mathfrak{p}_1$ and $\mathfrak{p}_2$). Both curves are in short Weierstrass form, and we use the remainder theorem on the pairs of $a_4$- and $a_6$-invariants separately to produce an SWF curve in $\mathcal{O}_K/(p)$ which reduces properly in  $\mathcal{O}_K/\mathfrak{p}_1$ and  $\mathcal{O}_K/\mathfrak{p}_2$ and therefore has the correct $a_{\mathfrak{p}_1}$-values and $a_{\mathfrak{p}_2}$-values.

\subsubsection{Restricted Models and Lifts}

Now that we have a list of valid SWF curves in $\mathcal{O}_K/(p)$, we may consider various restricted models of each curve in our attempt to find $E_{un}$. Restricted models have the form:

\

$\begin{array}{l}
E_{red}: \ \ y^2 + a_1xy + a_3y = x^3 + a_2x^2 + a_4x + a_6, \ \ \text{where:} \\
a_1,a_3 \in \{0,1,\varphi,\varphi+1\} \ \ \text{and} \\
a_2 \in \{0,\pm 1,\pm \varphi, \pm \varphi \pm 1\}
\end{array}$

\

\noindent By allowing $a_1$, $a_2$, and $a_3$ to be non-zero, there is perhaps a better chance for $a_4$ and $a_6$ to be smaller than they would be in a short Weierstrass model, making it more likely that we encounter a restricted model of $E_{un}$ itself using a fairly low-bound search.

In order to find the restricted models of our curves over $\mathcal{O}_K/(p)$, we can look at isomorphisms of the form $\tau=[r,s,t,1]$ which take each curve to one of the desired form. If $E: [a_1, a_2, a_3, a_4, a_6]$ is one of our SWF curves in $\mathcal{O}_K/(p)$, we know that $a_1 = a_2 = a_3 = 0$. Therefore, when looking at an isomorphism of the form $\tau$ from $E$ to a curve $E': [a_1', a_2', a_3', a_4', a_6']$ that we wish to be in restricted form, our expressions for $a_1'$ -- $a_3'$ simplify to:

\

$\begin{array}{rcllll}
a_1'&=&2s&&\Rightarrow2s\in\{0,1,\varphi,\varphi+1\}\\             
a_2'&=&3r-s^2&&\Rightarrow3r-s^2\in\{0,\pm 1, \pm \varphi, \pm \varphi \pm 1\}\\
a_3'&=&2t&&\Rightarrow2t\in\{0,1,\varphi,\varphi+1\}
\end{array}$

\

\noindent Since we are working in $\mathcal{O}_K/(p)$, we can choose $s$, $r$, and $t$ as follows (where $3^{-1}$ represents the inverse of 3 modulo $p$):

\

$\begin{array}{l}
s\in\{0,\frac{p+1}{2},\frac{p+1}{2}\varphi,\frac{p+1}{2}(\varphi+1)\} \\
\\
r\in\{3^{-1}s^2,3^{-1}(s^2\pm 1),3^{-1}(s^2\pm \varphi),3^{-1}(s^2\pm \varphi\pm 1)\} \\
\\
t\in\{0,\frac{p+1}{2},\frac{p+1}{2}\varphi,\frac{p+1}{2}(\varphi+1)\}
\end{array}$

\

\noindent These stipulations produce 144 different isomorphisms $\tau = [r,s,t,1]$ from $E$ to its restricted forms $E_{res}$. It is important to note that we are working in $\mathcal{O}_K/(p)$, so we must remember to reduce all coefficients modulo $p$. The exception is that we still want the components (i.e. the coefficients of the basis elements 1 and $a$) of $a_2$ to lie in the desired range, and reduction mod $p$ might not accomplish this (if $a_2=-1$, for example). The simple fix is that if reduction takes any component of $a_2$ to $p-1$, we subtract $p$ from this component.

We may now lift each of the 144 different curves from $\mathcal{O}_K/(p)$ to $\mathcal{O}_K$. One possible lift of a curve $E$ is to take the natural image of $E$ (i.e. leave each coefficient as it is, but consider them as elements of $\mathcal{O}_K$). Additionally, we can lift by adding or subtracting multiples of $p$ from any component of any coefficient. Since we want our lift to be in restricted form, $a_1$, $a_2$, and $a_3$ should maintain their values under any lift, so we only alter $a_4$ and $a_6$. In order to produce a reasonably-sized yet diverse collection of lifts to consider, we alter each in four ways (for $i \in \{4,6\}$):

\


$\begin{array}{rcl}
a_i & \mapsto & a_i \\
a_i & \mapsto & a_i - p \\
a_i & \mapsto & a_i - pa \\
a_i & \mapsto & a_i -p(\varphi+1)
\end{array}$

\

We therefore get 2304 restricted models in $\mathcal{O}_K$ with $-p \leq a_4[0], a_4[1],a_6[0],a_6[1] < p$, where the indices indicate which component of each coefficient is being considered. This gives us a decent breadth of curves to search through, while still limiting the coefficients to a reasonable range. We note that at no point in this process do we construct the curves, which is a relatively slow process in Sage. All manipulations to this point are on the coefficients, so for efficiency we store the ``curves'' as tuples. Once we have our lifts in $\mathcal{O}_K$, we use a custom function to quickly calculate the norm of the discriminant of each and check to see if the conductor norm of $E_{un}$ divides it. If so, we construct the curve and calculate its conductor to see if we have found $E_{un}$. As a final check, we calculate the $a_\mathfrak{p}$-values of the found curve to make sure that they match at all places with the $a_\mathfrak{p}$-values of $E_{un}$.

Considering restricted models is a reasonably effective method over $K = \QQ(\sqrt{5})$ because $\mathcal{O}_K$ has an infinite unit group. This means that any curve will have an infinite number of restricted forms: Given any curve, the transformation $\tau=[r,s,t,1]$ given above will produce a restricted form that has the same discriminant as the original curve ($\Delta = u^{12}\Delta ' = \Delta '$). Now, since $K$ has infinitely many units, we can apply a transformation where $u \neq 1$ to this model, and the discriminant of the resulting curve will necessarily be different from that of the original curve. This transformation might perturb the $a_1$-, $a_2$-, and $a_3$-invariants, but we can always apply a transformation to get this curve back into restricted form. This final model will be isomorphic to the original restricted form but will have different discriminant, so it must be a different model. This means that there is a good chance that our target curve $E_{un}$ will have a restricted model with relatively small coefficients, making our search more likely to encounter one of them, and therefore to uncover $E_{un}$.


\subsubsection{Looking at $\mathcal{O}_K/(p\cdot q)$}

The method described above is not guaranteed to find a given curve using only the primes we have considered (those with norm less than 100), and we see that if we want a more extensive search we have a couple of different options. First, we could extend the list of primes we work with. However, for large primes $p$, the number of ``valid" curves in $\mathcal{O}_K/(p)$ (i.e. those with the correct $a_{\mathfrak{p}_1}$ and $a_{\mathfrak{p}_2}$ values) quickly becomes unwieldy. The Sato-Tate distribution tells us that:
\begin{equation}
\#\{E/{\mathbb{F}_p} \ | \ \alpha \leq a_\mathfrak{p}(E) \leq \beta\} \approx \frac{1}{\pi}\int_{\alpha}^{\beta}\sqrt{4p-t^2} \ dt, \nonumber
\end{equation}

\noindent so unless our curve has decently \emph{large} $|a_\mathfrak{p}|$-values for the primes we add to our search, the process is going to be fairly slow. To illustrate this, for each of the following primes $p$ we have computed the number of curves over $\mathbb{F}_p$ which have $a_p$-value 0 (according to Sato-Tate, the most common value):

\

$\begin{array}{ccccccccccc}
\text{Prime (norm $<$ 100): \ }&11&19&29&31&41&59&61&71&79&89\\
\#\{E/{\mathbb{F}_p} \ | \ a_p(E) = 0\}:&20&36&84&90&160&348&180&490&390&528
\end{array}$

\

\noindent While not uniformly increasing, the number of curves indeed rises rapidly, and we must remember that for each of these curves we must compute all 2304 lifts described in $\S$4.2.3. Additionally, if we continue searching as before and increase the norm bound on our list of primes, we can still only be sure that each valid curve we consider has \emph{two} matching $a_\mathfrak{p}$ values with $E_{un}$, so the cost of adding larger primes with little added accuracy is likely not worth it.

Our second option is to use the Chinese Remainder Theorem again to produce curves over $\mathcal{O}_K/(p\cdot q)$ that have four correct $a_\mathfrak{p}$-values. The remainder theorem assures us that if we start with two curves obtained by the method in $\S$4.2.2, the resulting curve in $\mathcal{O}_K/(p\cdot q)$ will reduce correctly modulo $p$ and $q$, and therefore modulo $\mathfrak{p}_i$ and $\mathfrak{q}_i$ as well. So while $p\cdot q$ will be fairly large, we have introduced two new congruence conditions to give us a more accurate set of curves to consider. We can then perform lifts to $\mathcal{O}_K$ to search for reduced models of $E_{un}$ with $-p\cdot q \leq a_4[0], a_4[1], a_6[0], a_6[1] \leq p\cdot q$, which gives us great breadth and increased accuracy. Indeed, when we used this method to find curves that were missed by the simpler search which lifted directly from $\mathcal{O}_K/(p)$ to $\mathcal{O}_K$, every curve was found on the first iteration (i.e. the first product $p\cdot q$ such that $E_{un}$ has good reduction at the primes over both $p$ and $q$).

\subsubsection{Brief Summary of Results}

Altogether, we found 42 curves with norm conductor less than 1000 using the methods outlined above. By coding the algorithm described in $\S$4.2.1 through $\S$4.2.3 in Cython, we were able to find 37 of these curves in 28 minutes. The remaining five curves were found in 52 minutes using the modification described in $\S$4.2.4.

\subsubsection{Step-by-Step Algorithm to find $E_{un}$}

Here we give a summary of the methods used:

1. Factor $p = \mathfrak{p}_1 \cdot \mathfrak{p}_2$

2. Find all curves in $\mathcal{O}_K/\mathfrak{p}_1$ and $\mathcal{O}_K/\mathfrak{p}_2$ that have the correct $a_{\mathfrak{p}_1}$- and $a_{\mathfrak{p}_2}$-values

3. Produce curves in $\mathcal{O}_K/(p)$ using the Chinese Remainder Theorem

5. Lift the restricted models of these curves to $\mathcal{O}_K$ in various ways

6. If any of these lifts have the same conductor as $E_{un}$, check against all $a_\mathfrak{p}$-values

7. If all $a_\mathfrak{p}$-values match, this is $E_{un}$

8. If $E_{un}$ was not found, repeat steps 1-3 for two different split primes $p$ and $q$

9. Use the Chinese Remainder Theorem to produce curves in $\mathcal{O}_K/(p\cdot q)$

10. Repeat steps 5-7

\subsection{Integral Points (Cremona-Lingham)}
    As another approach to finding curves, we have used the method described by John Cremona and Mark Lingham in \cite{Cre-Lin}. The goal of this technique is to find all curves with good reduction at primes outside of a finite set $\mathcal{S}$ of primes in $K$ (in our case, $K = \mathbb{Q}(\sqrt{5})$). A $\textsc{Magma}$ implementation of the described algorithm (which has intrinsic limitations---especially over general number fields---due to the difficulty of finding integral points on elliptic curves) has been provided by Cremona and has proven effective to a certain extent. A concise summary of this method is given in a review of \cite{Cre-Lin} on MathSciNet (see \cite{msn-review}), and the main idea of the algorithm is as follows:

\begin{enumerate}

\item Compute the finite $m$-Selmer groups $K(\mathcal{S},m)$ of $K^*$, where
\begin{equation}
    K(\mathcal{S},m) = \{x\in K^*/(K^*)^m \ | \ ord_\mathfrak{p}(x) \equiv 0 \ (mod \ m) \ \ \forall \mathfrak{p}\notin \mathcal{S}\} \nonumber
\end{equation}
The algorithm requires the computation of these groups for $m = 2,3,4,6,\text{and}\ 12$.

\item From these $m$-Selmer groups, compute a finite set of possible $j$-invariants such that each elliptic curve with good reduction outside $\mathcal{S}$ has $j$-invariant in this set. These $j$-values are either $j=0,1728$, cases which can be treated directly, or $\mathcal{S}$-integers in $K$ satisfying $w \equiv j^2(j-1728)^3 \ (\text{mod} \ K^{*6})$ for $w$ in a specific subgroup of $K(\mathcal{S},6)$. In the latter case $j$ is of the form $j = \frac{x^3}{w}$, where $(x,y)$ is an $\mathcal{S}$-integral point on the elliptic curve $E_w: Y^2 = X^3 - 1728w$.

\item From this set of $j$-invariants, construct each curve with the desired reduction properties (indeed, there are finitely many by Shafarevich's Theorem). We must also check that each curve found has good reduction at the primes above 2 and 3 (if these primes are not in $\mathcal{S}$).

\end{enumerate}

The issue of finding all $\mathcal{S}$-integral points over $K$ (which give the $j$-invariants in step 2) presents a problem. For example, finding the generators of a curve's Mordell-Weil group---one technique for finding integral points---can be difficult if these generators have large coefficients (this makes them hard to find with reasonable search bounds) or when the curve's rank is high (in this case, there are multiple generators to find). Due to these limitations, Cremona uses a naive search to directly find $\mathcal{S}$-integral points within a given search region. With a larger bound, the program will find more curves but will run slower than with a smaller bound. Note that searches such as these will generally run much faster over $\mathbb{Q}$ than over number fields, so this method is less effective over $K$. Indeed, we see these limitations in the output of the algorithm---it failed to find a good number of curves, including one of conductor $-29\varphi+10$, which we knew to exist by looking at corresponding modular forms and was found by the method described in $\S$4.2.4 (the $a$-invariants are $[0, \varphi + 1, \varphi, 25857\varphi - 41835, 2396223\varphi - 3877170]$).

Despite the disadvantage of working in an extension of $\mathbb{Q}$, the Cremona-Lingham method had some success. Altogether, it found 36 previously unknown curves with norm conductor less than 2000, including some with very large $a$-invariants, which would be virtually impossible to find using a straightforward naive search. For example, the method found the curve $E$ with $a$-invariants $[0, 0, 0, -4122575271\varphi - 2547891639, -152431815268008\varphi - 94208042802474]$. It is important to note that while these coefficients are very large, the method found the curve by finding a point $(x,y)$ on its corresponding $E_w$ with relatively \emph{small} coefficients. Indeed, by factoring the $j$-invariant of $E$ we see:
\begin{equation}
-\frac{87325496}{121}\varphi - \frac{54204053}{121} \ \ = \ \ j_E \ \ = \ \ \frac{x^3}{w} \ \ = \ \ \frac{(27\varphi + 178)^3}{968\varphi - 1573} \nonumber
\end{equation}
When we solve for $y$ we see that the point found is $(x,y) = (27\varphi+178,216\varphi + 2953)$, which is certainly reasonable.

\subsection{Torsion Families}
    In the preliminary stages of our search, two methods that proved effective to a certain extent were searching through families of curves with given torsion structures and using quadratic twists on curves already known. While is is unlikely that these techniques would have produced a huge number of curves with reasonable search bounds, their implementation was both instructive and somewhat successful---together they produced 25 of the missing curves. The next two sections will briefly describe torsion families and twisting twisting in more detail and give a summary of how they were used to find new curves. For proofs of the facts presented, consult Ian Connell's \emph{Elliptic Curve Handbook}, which can be found online at  http://www.ucm.es/BUCM/mat/doc8354.pdf. Connell's book, Silverman's \emph{The Arithmetic of Elliptic Curves}, and Kubert's \emph{Universal Bounds on the Torsion of Elliptic Curves} are the primary resources used throughout [FIX REFERENCES].

\

We know that for any elliptic curve $E/K$,
\begin{equation}
E(K) \cong \mathbb{Z}^r \times E(K)_{tors}, \nonumber
\end{equation}
and the structure of the torsion subgroup $E(K)_{tors}$ is an important characteristic of $E$. Sheldon Kamienny and Filip Najman have shown that over $K = \mathbb{Q}(\sqrt{5})$, the torsion subgroup will be isomorphic to one of the 16 groups below, with the last case appearing only once:

\

$\begin{array}{lll}
\mathbb{Z}/m\mathbb{Z},   &1 \leq m \leq 10,& m = 12,\\
\mathbb{Z}/2\mathbb{Z} \oplus \mathbb{Z}/2m\mathbb{Z}, &  1 \leq m \leq 4,&\\
\mathbb{Z}/15\mathbb{Z}.&&
\end{array}$

\

Over $\mathbb{Q}$, these families have explicit parametrizations (given on page 217 of Daniel Kubert's \emph{Universal Bounds on the Torsion of Elliptic Curves}), which take in either one or two elements in $\mathbb{Z}$ and produce a curve with the given torsion structure. Over $\mathbb{Q}(\sqrt{5})$ though, these parametrizations may not be guaranteed to produce all curves with the given structure, but they are a good place to start when searching for such curves.

The idea of our implementation of the torsion family method was to essentially run a naive search up to a certain bound on the various parametrizations, compare the output to our list of known curves, and identify new curves accordingly.  However, we refined this method in some places by using the $a_p$ values for our unknown curves in conjunction with our code to compute isogeny degrees to identify unknown curves $E_{un}$ with given $p$-isogenies. These curves can be likely candidates for curves with $p$-torsion points, because $p$-isogenies are sometimes obtained by looking at a curve modulo $p$-torsion points (in the case $p=2$, these \emph{are} the curves with $p$-torsion points). Therefore, we can use the parametrizations of curves with given torsion structures in an attempt to find curves that have the known norm conductor of $E_{un}$. With luck, one of these curves will $E_{un}$ exactly.

We also used code that calculates the predicted torsion structures for each of our missing curves. These predictions were helpful, as they gave us an indication of which parametrizations might be more fruitful than others. For example, it helped us to find such curves as:
\begin{equation}
E:  y^2+xy=x^3+(121\varphi-211)x+619\varphi-1006, \nonumber
\end{equation}
which has a 6-torsion point.

Overall, the search was fairly successful for the 2-, 3-, 5-, and 7-torsion families, and was eventually extended to accommodate the rest of the possible torsion structures listed above. However, this search was quite slow because it depended on nested for-loops to search all possible parametrizations up to some bound. Also, over $K$, the number of parameters we must provide to each parametrization is twice the number we must provide over $\mathbb{Q}$ because each element in $\mathcal{O}_K$ has two components, so this significantly slowed the method as well. A major advantage to this search, though, is that with relatively small inputs, the parametrizations are able to produce curves with large and diverse $a$-invariants---curves that a regular naive search would certainly miss.

\

\subsection{Quadratic Twists (and a bit of background on isomorphisms)}

If $E$ is an elliptic curve defined over a number field $K$ (in our case $\mathbb{Q}$($\sqrt{5}$)), then a \emph{twist} $E'$ of $E$ is a curve isomorphic to $E$ over some extension of $K$. One way to create such isomorphisms is by defining four-parameter maps: $\tau$ $= [r,s,t,u]$, where $r,s,t \in K$ and $u \in K^*$. These maps act on the space $\mathcal{E}$ of non-singular elliptic curves by: 
\begin{equation} \tau: [a_1, a_2, a_3, a_4, a_6] \mapsto [a_1', a_2', a_3', a_4', a_6'], \nonumber\end{equation}
where:

$\begin{array}{rrrrcl}
&&&a_1' &= & u^{-1}(a_1+2s), \nonumber \\
&&&a_2' &= &u^{-2}(a_2-sa_1+3r-s^2),\nonumber \\
&&&a_3' &= &u^{-3}(a_3+ra_1+2t),\nonumber\\
&&&a_4' &= &u^{-4}(a_4-sa_3+2ra_2-(t+rs)a_1+3r^2-2st),\nonumber\\
&&&a_6' &= &u^{-6}(a_6+ra_4+r^2a_2+r^3-ta_3-t^2-rta_1)\nonumber
\end{array}$

\

\noindent The set of these maps $\tau$ form a group ($G$) under composition, and all isomorphisms between curves over $K$ may be given by elements in $G$. Therefore, two elliptic curves are isomorphic iff they are in the same $G$-orbit of $\mathcal{E}$. Conventionally, a $G$-orbit of $\mathcal{E}$ is called an \emph{abstract elliptic curve}, while a specific curve within an orbit is referred to as a \emph{model}. It is worth noting that there is an explicit formula for the image of a given point $(c, d)$ $\in$ $E$ under the action of the map $\tau$ $\in$ $G$:
\begin{equation} \tau: (c, d) \mapsto (u^{-2}(c-r), u^{-3}(d-sc+sr-t)) \nonumber\end{equation}
For example, the map $[-1]$, which sends every point on $E$ to its negative, is given by the map $\sigma$: $[0,-a_1,-a_3,-1]$ (quick sanity check: we see that if our curve is in short Weierstrass form (in particular $a_1 = a_3 = 0$), $\sigma$ indeed maps a point $(c, d)$ to its negative $(c, -d)$). We also note that the \emph{degree} of an isomorphism of curves $\tau = [r,s,t,u]$ is equal to $[K(r,s,t,u) : K]$. A \emph{quadratic twist} of a curve $E$ is therefore a curve $E'$ isomorphic to $E$ by a map of degree 1 or 2. If we consider $E$ in the form
\begin{equation}
E:   y^2 = x^3 + ax^2 + bx + c, \nonumber
\end{equation}
then the quadratic twist of E \emph{by} $d$, for any element $d \in K^*$, is given by the equation
\begin{equation}
E^d:   dy^2 = x^3 + ax^2 + bx + c.
\end{equation}
We can easily transform $(1)$ to Weierstrass form by multiplying through by $d^3$ and making the substitutions $y' = d^2y$ and $x' = dx$. Finally, if $E: [a_1, a_2, a_3, a_4, a_6]$ is our curve, we can give the isomorphism from $E$ to $E^d$ by the map $\tau_d = [0, a_1(\frac{r-1}{2}), a_3(\frac{r^3-1}{2}), r]$, where r is a root of $r^2 = \frac{1}{d}$. In fact the map from $K^*$ to $G$ given by $d \mapsto \tau_d$ is a homomorphism.

Our implementation of twisting to find new elliptic curves is fairly straightforward: loop through values of $i$ and $j$ up to a certain bound and twist the inputted curve $E$ by $i\varphi+j$, where $\varphi = \frac{1+\sqrt{5}}{2}$ (which generates $\mathcal{O}_K$). This method has similar advantages and disadvantages to the torsion family method; the advantage of this system is that the result of a quadratic twist by $i\varphi+j$ can have vastly different coefficients than the original curve, including \emph{much} larger values, and a significant disadvantage is that we are still looping through values, which is very slow, especially since we must check each potential new curve against our list of known curves to see if it is isomorphic or isogenous to any curve of the same norm conductor. However, there is an intuitive bound on how high the norm of $d$ should be (if we are twisting by $d$): the norm conductor of $E_d$ (denote as $||cond(E_d)||$) is divisible by $||d\cdot cond(E)||$, so we need not consider twists by elements of norm too large for our table of curves with $N \leq 1000$.

\

\subsection{Using Special Values of $L$-functions and Periods}
    \newcommand{\n}{\mathfrak{n}}
\newcommand{\ap}[1]{a_{\p_{#1}}}
\newcommand{\round}[1]{\left\lfloor{#1}\right\rceil}
\newcommand{\fc}{\mathfrak{c}}

In (INSERT REFERENCE), Lassina Dembele outlines a method for finding modular elliptic
curves from Hilbert modular forms over totally real fields. In the case of nonsquare level
the method relies on computing or guessing periods of the curve from L-functions of twists
of the curve evaluated at 1. In particular, the only inputs required are the level of the
Hilbert modular form and its L-series. So we suppose that we know the level $\n = (N)$ of the form,
where $N$ is totally positive,
and that we have sufficiently many coefficients of its L-series $\ap{1}, \ap{2}, \ap{3}, \ldots$.

Associated to an elliptic curve is a period lattice \ldots (explain something here?)

Suppose $E$ is an elliptic curve over $\Q(\sqrt 5)$. It does not make sense to speak of
\emph{the} period lattice of $E$, as there are multiple embeddings of $E$ into the complex
numbers. So let $\alpha_1$ and $\alpha_2$ denote the embeddings of $K$ into the real numbers,
for concreteness with $\alpha_1$ chosen so that $\alpha_1(\phi) > 0$. Then we get two period
lattices $L_{\alpha_1}(E)$ and $L_{\alpha 2}(E)$, corresponding to the two different embeddings
of $K$ in $\R$. For convenience, we will henceforth choose $\alpha_1$ as a distinguished embedding;
we think of $L_{\alpha_1}(E)$ as the period lattice of $E$ and $L_{\alpha_2}(E)$ as the period lattice
of $\bar E$. (Note that $L_{\alpha_1}(\bar E) = L_{\alpha_2}(E)$.)

\newcommand{\Omegap}{\Omega^+}
\newcommand{\Omegam}{\Omega^-}
\newcommand{\Omegapp}{\Omega^{++}}
\newcommand{\Omegapm}{\Omega^{+-}}
\newcommand{\Omegamp}{\Omega^{-+}}
\newcommand{\Omegamm}{\Omega^{--}}

So, with this understanding, let $\Omegap_E$ denote the least real period of $E$, and let
$\Omegam_E$ denote the least imaginary period of $E$, a purely imaginary number; thus, we either
have $L_{\alpha_1}(E) = \Omegap\Z + \Omegam\Z$ or $L_{\alpha_1}(E) = \Omegap\Z + 1/2(\Omegap + \Omegam)\Z$.
Furthermore,
as periods will often occur mixed together in pairs, we define
\begin{eqnarray*}
\Omegapp_E &=& \Omegap_E\Omegap_{\bar E} \\
\Omegapm_E &=& \Omegap_E\Omegam_{\bar E} \\
\Omegamp_E &=& \Omegam_E\Omegap_{\bar E} \\
\Omegamm_E &=& \Omegam_E\Omegam_{\bar E}.
\end{eqnarray*}
We will refer to these numbers as the mixed periods $E$.

Dembele's method relies on two observations. First, if we know these four numbers (or actually just the first three,
but from any three of them we can recover the fourth), we can recover the curve
$E$, and second, we can compute these numbers, or at least integer multiples of these numbers, by computing
central values of twists of the L-function associated to $E$.

\subsubsection{Recovering the curve from its mixed periods}

First, let us assume that we know the mixed periods of the curve to infinite precision. Absent the knowledge
of the discriminant of the curve, we do not precisely know the the lattice type of the curve and its conjugate,
but there are only a few possibilities for what they may be. So we compute the $4$ complex numbers
\begin{eqnarray*}
    \tau_1(E) &=& \frac{\Omegamp_E}{\Omegapp_E} = \frac{\Omegam_E}{\Omegap_E} \\
    \tau_2(E) &=& \frac{1}{2}\left(1 + \frac{\Omegamp_E}{\Omegapp_E}\right) = \frac{1}{2}\left(1 + \frac{\Omegam_E}{\Omegap_E}\right) \\
    \tau_1(\bar E) &=& \frac{\Omegapm_E}{\Omegapp_E} = \frac{\Omegam_E}{\Omegap_E} \\
    \tau_2(\bar E) &=& \frac{1}{2}\left(1 + \frac{\Omegapm_E}{\Omegapp_E}\right) = \frac{1}{2}\left(1 + \frac{\Omegam_{\bar E}}{\Omegap_{\bar E}}\right).
\end{eqnarray*}
We now compute $j(\tau)$ for each of numbers above, where (letting $q = \exp(2 \pi i \tau)$)
\[
    j(\tau) = \frac{1}{q} + 744 + 196884q + 21493760q^{2} + \cdots
\]
is the modular $j$-function. (In practice, we want to normalize $\tau$ to be in the standard fundamental
domain for the action of $SL(2, \Z)$, so that the Fourier series for $j(\tau)$ converges nicely.) This should
give us $4$ real numbers $j_1(E), j_2(E), j_1(\bar E),$ and $j_2(\bar E)$, and we will have
$\sigma_1(j(E)) = j_{(1 \textrm{ or } 2)}(E)$ and $\sigma_2(j(E)) = j_{(1 \textrm{ or } 2)}(\bar E)$. We try
each of the four possibilities in turn, recognizing $j(E)$ as an algebraic number from its two embeddings and
constructing a curve with the desired $j$-invariant. Once we have a curve we can examine its Fourier coefficients
to determine whether or not it is a twist of the curve we are looking for, and, if it is, we determine
which twist it is.

In practice, of course, we only have limited precision, and as $j(E)$ will not be an algebraic integer it may
not be feasible to directly determine it exactly, especially if it has a rather large denominator. Still,
the above approach may be useful.

To get around issues of limited precision, we suppose that we have some extra information; namely, the discriminant
$\Delta_E$ of the curve we are looking for. (In practice we start by guessing that $\Delta_E = N$, and then we
introduce unit factors and increment powers of the factors of $N$ until we find the curve that we are looking for.)
With $\Delta_E$ in hand we can directly determine which $\tau$ to choose: if $\sigma_1(\Delta_E) > 0$ then
$\sigma_1(j(E)) = j(\tau_1(E))$, and if $\sigma_1(\Delta_E) < 0$ then $\sigma_1(j(E)) = j(\tau_2(E))$, and
similarly for $\sigma_2$. We now compute $\sigma_1(c_4(E)) = (j(\tau) \sigma_1(\Delta_E))^{1/3}$ and
$\sigma_2(c_4(E)) = (j(\tau') \sigma_2(\Delta_E))^{1/3}$.

With approximations of the two embeddings of $c_4$ in hand we can recognize $c_4$ approximately as an algebraic
integer. Specifically, we compute
\[
    \alpha = \frac{1}{2}\round{\sigma_1(c_4) + \sigma_2(c_4)}
\]
and
\[
    \beta = \frac{1}{2}\round{\frac{\sigma_1(c_4) - \sigma_2(c_4)}{\sqrt{5}}},
\]
where $\round(x)$ denotes the nearest integer function. We then set $C' = \alpha + \beta\sqrt{5}$. This may
not actually be an integer, so we convert it to $C = a + b \varphi$, arbitrarily rounding either $a$ or $b$
in the case that $C'$ was not actually integer. Now we vary $m$ and $n$ in some range around $0$ and for
each of the possibilities
\[
    c_{4, \mathrm{guess}} = (a + m) + (b + n)\phi
\]
we attempt to solve
\[
    c_{6, \mathrm{guess}} = \pm \sqrt{c_{4, \mathrm{guess}}^3 - 1728 \Delta_E}.
\]
Each time this has a solution we construct a curve $E_{\mathrm{guess}}$ with given $c_4$ and $c_6$, check
if it has the correct conductor, and, if so, check if its Fourier coefficients are the ones that we are looking
for. If it passes these checks, we declare that $E = E_{\mathrm{guess}}$.

\subsubsection{Dirichlet characters and twists}

To compute the mixed periods of the curve we will need to compute central values of the $L$-function
twisted by quadratic Dirichlet characters over $O_K$. Specifically, such a character $\chi$ is a completely
multiplicative function $O_K \rightarrow {-1, 0, 1}$ that is periodic with respect to some minimal ideal $\a$,
so that $\chi(x)$ = $\chi(n + a)$ for all $n \in O_K$ and all $a \in \a$. Equivalently, we may consider a
homomorphism $\chi : (O_K/\a)^\cross \rightarrow {\pm 1}$ and extend it to $O_K$ by lifting and defining
$\chi(n)$ to be $0$ whenever $x$ is not relatively prime to $\a$.

For any $\a$ there is a \emph{trivial} character $\chi_0(n)$, which is the function that takes the value $1$
whenever $n$ is relatively prime to $\a$ and $0$ otherwise. We say a character $\chi$ is imprimitive if its
least period is $\a$ and it can be written as $\chi(n) = \chi_0(n)\chi_1(n)$, where $\chi_1(n)$ is a character
with a smaller period than $\chi$. If $\chi_1$ has smallest possible period for
such a representation, we call the period of $\chi_1$ the conductor of $\chi$. If the conductor of $\chi$ is
also the modulus of $\chi$, then we say that $\chi$ is primitive. These are the characters that we are
interested in.

For simplicity, we compute only with characters of prime modulus. For each prime modulus $\p$ there is
exactly one quadratic character, and it is primitive. To construct it and compute with it, one can simply
construct $O_K/\p$, which will be a finite field, and will thus have a multiplicative generator. We choose
such a generator and assign it the value $-1$; this completely determines the values of $\chi$. We can
construct a table of these values and use it for evaluation of $\chi$.

The use of $\chi$ comes in twisting the $L$-function of $E$ by $\chi$ to compute central values
which are related to the mixed periods of $E$. If we have
\[
    L(E, s) = \sum_{\m \subseteq \cO_K} \frac{a_\m}{N(\m)^s},
\]
then for primitive $\chi$ with conductor relatively prime to the conductor of $E$, the twisted $L$-function
is given by
\[
    L(E, \chi, s) = \sum_{\m \subseteq \cO_K} \frac{\chi(m) a_\m}{N(\m)^s},
\]
where $m$ is a totally positive generator of $\m$. The functional equation satisfied by $L(E, \chi, s)$ is
the same as that of $L(E, s)$ except the conductor is multiplied by the square of the norm of the conductor
of $\chi$ and the sign is multiplied by $\chi(-N)$.

\subsubsection{Finding the mixed periods}
The key to finding the periods of $E$ is the following conjecture of SOMEONE. (Is this a natural generalization
of something that is known over $\Q$?)
\begin{conjecture} If $\chi$ is primitive with conductor $\fc$ relatively prime to the conductor of $E$,
with $\chi(\varphi) = s'$ and $\chi(1 - \varphi) = s$, $s, s' \in \{+, -\} = \{+1, -1\}$, then
\[
    \Omega^{s,s'}_E = c_\chi i^{ss'} \sqrt{5} \sqrt{(2, N(\fc))N(\fc)} L(E, \chi, 1)
\]
for some integer $c_\chi$.
\end{conjecture}
\begin{remark}
I've specialized the conjecture to $\Q(\sqrt 5)$ and I've slightly modified the it to include what I believe to be
the absolute value of the Gauss sum, and its sign up to $\pm 1$. Perhaps I need to look at this a little bit more
carefully.
\end{remark}

With this conjecture in place we can now fairly easily compute integer multiples of each of the mixed periods of
$E$ when the conductor is not a square, and with some extra work sometimes we even recover the periods directly.
When the conductor is a square, there is a problem, however. The sign of $L(E, \chi, s)$ is given by
$\epsilon_E \chi(-N),$ which is the same as $\epsilon_E \chi(-1)$ when $N$ is a square. $\chi(-1)$ is determined
by $\chi(-1) = \chi(\varphi)\chi(1 - \varphi)$, so if $\epsilon_E = 1$, then $L(E, \chi, 1)$ will only be 
nonvanishing when $\chi(\varphi) = \chi(1 - \varphi)$, and we can only obtain information about $\Omegapp$ and
$\Omegamm$. Similarly, when $\epsilon_E = -1$ and $N$ is a square we can only obtain information about
$\Omegapm$ and $\Omegamp$.

Suppose then that $N$ is not a square, and again assume that we can compute with infinite precision, or
at least all the precision we like. Then to find the mixed periods we choose some bound $M$ on the conductors
of the characters we are willing to consider and first make four lists of characters
\[
    S^{s,s'} = \{ \chi \bmod \p : \chi(\phi) = s', \chi(1 - \phi) = s, (\p, \n) = 1, N(\p) < M, \chi(-N) = \epsilon_E\}
\]
Here $s, s' \in \{\pm 1\} = \{+, -\}$. We will consider these lists to be ordered by the norm conductors of the
characters in increasing order, and index their elements as $\chi^{s,s'}_0, \chi^{s,s'}_1, \chi^{s,s'}_2, \ldots$.
For each character we compute the central value of the twisted $L$-function to get four new lists
\[
    \mathcal{L}^{s,s'} = \{ i^{ss'}\sqrt{5 (2, N(\p)), N(\p)} L(E,\chi,1), \chi \in S^{s,s'}\} =
        \{\mathcal{L}^{s,s'}_0, \mathcal{L}^{s,s'}_1, \ldots\}
\]
These should all be integer multiples of the mixed periods that we are looking for. To get a good idea of which
multiples we should guess they are, we compute each of the ratios
\[
    \frac{\mathcal{L}^{s,s'}_0}{\mathcal{L}^{s,s'}_k} = \frac{c_{\chi^{s,s'}_0}}{c_{\chi^{s,s'}_k}} \in \Q, k = 1, 2, \ldots
\]
and recognize these as rational numbers (ignoring precision issues for the moment). If it turns out that
any pair $c_{\chi^{s,s'}_k},c_{\chi^{s,s'}_l}$ is relatively prime, will be able to determine $\Omega^{ss'}$
exactly. Regardless of this, we choose as an initial guess
\[
    \Omega^{ss'}_{E, \mathrm{guess}} = \frac{\mathcal{L}^{s,s'}_0}{M^{s,s'}}
\]
where
\[
    M^{s,s'} = \mathrm{lcm}\left\{ \mathrm{numerator}\left(\frac{\mathcal{L}^{s,s'}_0}{\mathcal{L}^{s,s'}_k}\right),
        k = 1,2, \ldots \right\}
\]

\subsubsection{Examples and putting things together}
The above discussion leaves some issues to be determined. In pratice we have only a limited number of coefficients
of the $L$-functions that we need to work with, so our precision is somewhat limited. Also, there are a number
of guesses that need to be made in the algorithm, so we need to have a good way of understanding how to make those
guesses. We start with a few examples, which may provide some enlightenment, and then proceed to make
suggestions for how a general search method should work.

\textbf{Example 1.} We start with a simple example, using the same curve a Dembele uses in his Example 1. So let
$E$ be the curve over $K = \Q( \sqrt 5)$ with $a$-invariants $[1, -(1+\varphi), \varphi, 0, 0]$, so
\[
    E := y^2 + xy + \varphi y = x^3 - (\varphi+1)x^2.
\]
Using Sage, we compute a basis for the period lattice of E (recall that we are setting a 
distinguished embedding) and find one that is (approximately)
\[
(3.05217315335726, 2.39884476932372i).
\]
With the other embedding of $K$ into $\RR$ we find a basis for the period lattice of $\bar E$ as
\[
(8.43805988789973, 4.21902994394986 + 1.57216678613265i).
\]
So we have
\begin{eqnarray*}
    \Omegapp_E &\approx& 3.05217315335726 \times 8.43805988789973 \approx 25.7544198562683 \\
    \Omegapm_E &\approx& 3.05217315335726 \times (2 \cdot 1.57216678613265 i) \approx 9.59705051446808i \\
    \Omegamp_E &\approx& 2.39884476932372i \times 8.43805988789973 \approx 20.2415958253286i \\
    \Omegamm_E &\approx& 2.39884476932372i \times (2 \cdot 1.57216678613265 i) \approx -7.54276814283759
\end{eqnarray*}
Note that here we have $c_4(E) = 16 \varphi + 25$ and $\Delta_E = 16 \varphi + 9$, from which we can compute
$j(E) = c_4(E)^3/\Delta_E = \frac{1}{31}(106208\varphi - 54753)$, and find that
\[
    \sigma_1(j(E)) \approx 3777.26302829512, \ \ \  \sigma_2(j(E)) \approx -3883.65012506932.
\]
We can also compute these numbers from the mixed periods of $E$. Since $\sigma_1(\Delta) > 0$, we have
\[
    \sigma_1(j(E)) = j(\Omegamp_E/\Omegapp_E) \approx j(0.785946487565785i)
\]
If we compute this directly using $10$ terms in the Fourier expansion of the $j$-function, we will find that
\[
    j(0.785946487565785i) \approx 3777.26301983227,
\]
which is not too far from the actual $j$-invariant. We can do better however. Under the action of $\SL(2, \Z)$,
$0.785946487565785i$ maps to $1.27235125523263i$, and if we compute the $j$-function here using ten terms
we will find that
\[
    j(1.27235125523263i) \approx 3777.26302829512,
\]
which completely agrees with the $j$-invariant computed above. Similarly, since $\sigma_2(\Delta) < 0$,
we can compute that
\[
    \sigma_2(j(E)) = j\left( \frac{1}{2}\left(1 + \Omegamm_E/\Omegapm_E\right)\right)
            \approx j(0.5 + 0.186318514803048i).
\]
In this case if we try to compute the with ten terms in the Fourier expansion at $53$ bits of precision
we will find that
\[
     j(0.5 + 0.186318514803048i) \approx -5.93248167367531\cdot 10^{10} + 0.0000675970580457112i,
\]
which is complete nonsense. We can do better by using something like $200$ terms and $200$ bits of precision, which will give
\[
    j(0.5 + 0.186318514803048i) \approx -3883.6501250693112
\]
or we just translate $0.5 + 0.186318514803048i$ to $0.5 + 1.34178828263132$, use ten terms at $53$ bits of
precision, and find that
\[
    j(0.5 + 1.34178828263132) \approx -3883.65012506932.
\]
With this much precision for the two embeddings of $j(E)$, it is somewhat possible to recover $j(E)$ as
an algebraic number, though we really only would have such confidence because we already know what $j(E)$ is.
Set $j_1 = \sigma_1(j(E))$ and $j_2 = \sigma_2(j(E))$. If we compute the continued fraction expansion of
$.5(j_1 + j_2)$, for example we will get the sequence of convergents
\[
    -54, -53, -\frac{266}{5}, -\frac{1383}{26}, -\frac{1649}{31},
\]
and the last one is in fact exactly $2\tr(j(E))$ However, it seems recognizing $j(E)$ directly this
way would be fraught with precision issues. Instead we can use our knowledge of $\Delta_E$ to work
backwards, and find that
\[
    \sigma_1(c_4) = \sigma_1(\Delta_E) \approx 50.8885438199983
\]
and
\[
    \sigma_2(c_4) = j_2 \sigma_2(\Delta_E) \approx 15.1114561800017.
\]
With these we can easily recognize $c_4$ as an algebraic integer. We compute
\[
    \tr(c_4) = 66
\]
and
\[
    \tr(c_4/\sqrt{5}) = (\sigma_1(c4) - \sigma_2(c4))/\sigma_1(\sqrt{5}) = 16,
\]
so $c_4 = 33 + 8\sqrt{5} = 33 + 8(2 \varphi - 1) = 25 + 16 \varphi$. We solve for
$c_6 = \pm \sqrt(1728\Delta_E - c_4^3)$. There are two possibilities for $c_6$, so we try each
of the two curves, and find that one of them is the curve that we are looking for.

We still have not dealt with the issue of how we might find the periods of the curve without knowing
the curve. In this case the functional equation for $L(E, s)$ has sign $+1$, so we will twist by characters
$\chi$ such that $\chi(-N) = 1$. So we look through the prime ideals of $O_K$ in order of norm
and we find that the characters with conductors $(3), (3 \varphi - 1), (3 \varphi - 2)$ and $(\varphi + 5)$
suit our purposes. With these choices, we have
\begin{eqnarray*}
    \chi_{(3)}(\phi) = -1, \ \  && \ \ \chi_{(3)}(1 - \varphi) = -1 \\
    \chi_{(3\varphi - 1)}(\phi) = 1, \ \  && \ \ \chi_{(3\varphi - 1)}(1 - \varphi) = -1 \\
    \chi_{(3\varphi - 2)}(\phi) = -1, \ \  && \ \ \chi_{(3\varphi - 2)}(1 - \varphi) = 1 \\
    \chi_{(\varphi + 1)}(\phi) = 1, \ \  && \ \ \chi_{(\varphi + 2)}(1 - \varphi) = 1.
\end{eqnarray*}
Thus, for example, $L(E, \chi_{(3)}, 1) \sqrt{5 N(3)}$ should be a multiple of $\Omegamm_E$. Indeed, using all
of the primes of norm up to $40000$ we compute $ L(E, \chi_{(3)}, 1) \approx 1.12440948706034$, and
$3 \sqrt {5} \times 1.12440948706034 \approx 7.54276814283777$, which we recognize as very close to
$\Omegamm_E$.

\textbf{Example 2.} We now proceed to a more complicated example. Suppose that we have found $a_\p$ for all
primes of norm less than $40000$ for a Hilbert modular form over $K$ of level $30*a - 10$, with $\epsilon_f = 1$.
We examine the characters with conductors up to $1500$ and taking at most $10$ of each type that we
are looking for, we build the lists

\begin{multline}
S^{--} = \big\{\chi_{(a + 6)}, \chi_{(7)}, \chi_{(7a - 3)}, \chi_{(a + 10)}, \chi_{(-11a + 7)}, \\
\chi_{(-14a + 5)}, \chi_{(-15a + 4)}, \chi_{(-15a + 11)}, \chi_{(15a - 8)}, \chi_{(-18a + 13)}\big\}
\end{multline}
\begin{multline}
S^{-+} = \big\{\chi_{(a - 9)}, \chi_{(-8a + 5)}, \chi_{(a - 12)}, \chi_{(2a + 11)}, \chi_{(-12a + 5)}, \\
\chi_{(3a + 13)}, \chi_{(13a - 6)}, \chi_{(a - 16)}, \chi_{(16a - 5)}, \chi_{(-3a - 17)}\big\}
\end{multline}
\begin{multline}
S^{+-} = \big\{\chi_{(5a - 3)}, \chi_{(7a - 2)}, \chi_{(-8a + 3)}, \chi_{(a + 11)}, \chi_{(-2a + 13)}, \\
\chi_{(2a + 13)}, \chi_{(13a - 7)}, \chi_{(a + 15)}, \chi_{(-2a + 17)}, \chi_{(-17a + 7)}\big\}
\end{multline}
\begin{multline}
S^{++} = \big\{\chi_{(-a + 6)}, \chi_{(9a - 4)}, \chi_{(a - 14)}, \chi_{(a + 13)}, \chi_{(-17a + 5)}, \\
\chi_{(a - 22)}, \chi_{(-3a + 25)}, \chi_{(4a + 25)}, \chi_{(-4a + 29)}, \chi_{(25a - 8)}\big\}
\end{multline}

For each of these characters we evaluate $L(f, \chi, 1)$ using lcalc, and multiply by the appropriate
factors to get the lists $\mathcal{L}^{s,s'}$ above. Note that lcalc will warn us
that we don't have enough coefficients for accurate evaluation. This is correct, and in particular the
accuracy of the numbers in the following lists drops off as we more futher along the lists. So we
get the following lists of approximate values.

\begin{multline}
\mathcal{L}^{--} = \big\{23.0447957401108,
23.0447957401090,
69.1343872203071,
\\69.1343870414420,
276.537547685983,
0,
69.1339752521539,
\\69.1347565067475,
92.1785740140933,
69.1381805217699\big\}
\end{multline}
\begin{multline}
\mathcal{L}^{-+} = \big\{13.8729392831410i,
13.8729392838598i,
115.607826237882i,
\\226.591340602765i,
41.6188067753373i,
221.967022543403i,
226.591333886957i,
\\0,
124.855687410063i,
226.589793508735i\big\}
\end{multline}
\begin{multline}
\mathcal{L}^{+-} = \big\{36.8547445634673i,
12.2849148547115i,
36.8547445497484i,
\\12.2849138747566i,
196.558637741332i,
147.418978438534i,
307.122767748292i,
\\36.8546922087215i,
110.564250431060i,
0\big\}
\end{multline}
\begin{multline}
\mathcal{L}^{++} = \big\{66.5595355493066,
66.5595355094869,
0,
\\66.5595717095333,
0,
599.042164153531,
266.230056409565,
\\266.076837348988,
66.5736728191717,
61.5776839596031\big\}
\end{multline}

Recall now that these numbers are all supposed to be integer multiples of the mixed periods of the
curve. The first number in each list is likely to be the most accurate, so we divide each list by the
first number and get a list of what should be rational numbers. We will use these to determine a good guess
for which integer to divide the first entry by.

\begin{multline}
 = \big\{1.00000000000000,
1.00000000000008,
0.333333333333455,
\\0.333333334195858,
0.0833333336935456,
0.333335319660977,
0.333331552818323,
\\0.250001651539841,
0.333315044830469\big\}
\end{multline}
\begin{multline}
 = \big\{1.00000000000000,
0.999999999948185,
0.120000001164239,
\\0.0612244900720257,
0.333333422027897,
0.0625000016857383,
0.0612244918866227,
\\0.111111792909987,
0.0612249080963401\big\}
\end{multline}
\begin{multline}
 = \big\{1.00000000000000,
2.99999999994569,
1.00000000037224,
\\3.00000023925259,
0.187499999933697,
0.249999999686837,
0.120000040484372,
\\1.00000142057205,
0.333333282862957\big\}
\end{multline}
\begin{multline}
 = \big\{1.00000000000000,
1.00000000059826,
0.999999456723869,
\\0.111109934378923,
0.250007592857631,
0.250151558521446,
0.999787644736028,
\\1.08090352331166\big\}
\end{multline}

Many of these are immediately recognizable as approximate rational numbers, while others take a little more effort.
A hueristic attempt to recognize each one automatically by examining the continued fraction convergents
and looking for big jumps in the denominators is likely to come up with lists of rational numbers that
look like the following.

\begin{equation*}
\big\{1,
1,
\frac{1}{3},
\frac{1}{3},
\frac{1}{12},
\frac{1}{3},
\frac{1}{3},
\frac{1}{4},
\frac{1}{3}\big\}
\end{equation*}
\begin{equation*}
\big\{1,
1,
\frac{3}{25},
\frac{3}{49},
\frac{1}{3},
\frac{1}{16},
\frac{3}{49},
\frac{1}{9},
\frac{3}{49}\big\}
\end{equation*}
\begin{equation*}
\big\{1,
3,
1,
3,
\frac{3}{16},
\frac{1}{4},
\frac{3}{25},
1,
\frac{1}{3}\big\}
\end{equation*}
\begin{equation*}
\big\{1,
1,
1,
\frac{1}{9},
\frac{1}{4},
\frac{1}{4},
1,
\frac{8632075}{7985981}\big\}
\end{equation*}
These lists look reasonable except for the last entry in the final list. To examine it more closely,
we look at the convergents of the continued fraction expansion of $1.08090352331166$, which look like
\begin{multline}
\big[1, \frac{13}{12}, \frac{27}{25}, \frac{40}{37}, \frac{147}{136}, \frac{334}{309}, \frac{1149}{1063},
\frac{1483}{1372}, \frac{2632}{2435}, \frac{12011}{11112}, \frac{14643}{13547}, \frac{26654}{24659},
\frac{41297}{38206}, \\ 
\frac{67951}{62865}, \frac{109248}{101071}, \frac{177199}{163936}, \frac{286447}{265007}, \frac{463646}{428943},
\frac{8168429}{7557038}, \frac{8632075}{7985981}, \\ \frac{34064654}{31514981}, \frac{42696729}{39500962},
\frac{162154841}{150017867}\big].
\end{multline}
From this we can get an idea of what went wrong. We do not have enough precision for the denominator of
the continued fraction expansion to jump significantly enough for us to correctly guess the correct fraction.
The correct fraction is likely to be one of the first few. We have
\[
    13/12 = 1.08333\ldots,
\]
\[
    27/25 = 1.08
\]
\[
    40/37 = 1.081081\ldots,
\]
all of which are quite close to the value we are looking for. We choose $27/25$ as our guess for the correct
fraction on the assumption that small primes and squares are most likely to appear as factors. (This is
a somewhat weak justification, and difficult to automate.)

Examining these denominators, we make the guesses
\begin{eqnarray*}
    \Omegamm_{E, \textrm{guess}} = \mathcal{L}^{--}_0 / 1 &=& 23.0447957401108/1 = -23.0447957401108 \\
    \Omegamp_{E, \textrm{guess}} = \mathcal{L}^{-+}_0 / 3 &=& 13.8729392831410i/3 = 4.62431309438033i \\
    \Omegapm_{E, \textrm{guess}} = \mathcal{L}^{+-}_0 / 3 &=& 36.8547445634673i/3 = 12.2849148544891i \\
    \Omegapp_{E, \textrm{guess}} = \mathcal{L}^{++}_0 / 27 &=& 66.5595355493066/27 = 2.46516798330765.
\end{eqnarray*}
The actual periods will satisfy
\[
    \frac{\Omegamm_E \Omegapp_E}{\Omegamp_E \Omegapm_E} = 1,
\]
while in our case we have
\[
    \frac{\Omegamm_{E,\text{guess}} \Omegapp_{E, \text{guess}}}{\Omegamp_{E, \text{guess}} \Omegapm_{E,\text{guess}}} = 1.00000000027151,
\]
which is an indication that our guess is reasonable. (This relation could also be used to recognize that we
should choose $27/25$ above, and is useful in making guesses in general.) With these we get a few possibilities
for $\tau_E$ and $\tau_{\overline E}$. Translating these into the fundamental domain we get
\begin{eqnarray}
    t_1(E) &=& 1.87586124989975i \\
    t_2(E) &=& -0.5 + 0.937930624949874i \\
    t_1(\overline E) &=& 4.98339867208796i \\
    t_1(\overline E) &=& -0.5 + 2.49169933604398i
\end{eqnarray}
and $4$ corresponding pairs of guesses for the embeddings of $j(E)$:
\[
(132195.763704925, 3.967003831510192 \cdot 10^{13})
\]
\[
(132195.763704925, -6.297671571246 \cdot 10^{6} )
\]
\[
(-15.100427892327, 3.967003831510192 \cdot 10^{13}
\]
\[
(-15.100427892327, -6.297671571246 \cdot 10^{6} )
\]
(New idea:) It is hard to recognize any of these pairs as the embeddings of an algebraic number because the
denominators involved are quite large. However, we know something about the denominator; namely, $j(E)N^m$
is actually an algebraic integer for some power of $m$. So if we have enough precision, we should be able
to test a few powers of $N$ and find this integer. Indeed, with the fourth pair of possibilities, we find
that
\[
     -15.100427892327 \sigma_1(N)^3 -6297671.571246 \sigma_2(N)^3 \approx 146415110000
\]
and
\[
     -15.100427892327 \sigma_1(N)^3/\sqrt{5} + 629767.571246 \sigma_2(N)^3/\sqrt{5} \approx -65479601000,
\]
which suggests as a possibility for $j(E) N^3$
\[
    146415110000/2 - 65479601000\sqrt{5}/2 = -65479601000\varphi + 105947355500.
\]
This would give
\[
    j(E) = \frac{3748625197}{1331} \varphi - \frac{1213084615}{2662}.
\]
(This is in fact the correct $j$-invariant, and we should be able to construct a curve with this $j$
and recognize it as a twist of the curve we are looking for\ldots)



\section{Computing all Isogeny Classes}

\section{Saturating Mordell-Weil Groups}

\section{Computing $L$-series}

\section{Modularity}

(Statement of the conjecture.  Evidence we give.   What is known in this case (see email from Richard Taylor).)

\section{Table 1: Curves with Norm Conductor up to $1000$}


The actual tables will be sideways (landscape).... and depending on
length maybe in another document.  This might just be a discussion of
the tables.

\section{Table 2: Mordell-Weil Generators}

\section{Table 3: Hecke Eigenvalues}

\section{Table 4: Birch and Swinnerton-Dyer Data}

\section{Table 5: Parametrization Degrees}

(We will remove this section if nobody finds an algorithm to compute
these.  I don't know if there is an algorithm.  Note that one
complication is that there is a different parameterization
corresponding to each prime that exactly divides the level, and there
is no parameteriziation in some cases, e.g., when the level is a
perfect square.)



\bibliography{biblio}
\end{document}
